\section{Related Work} \label{relatedwork}

This section will present the state of the art related to this work. Since no relevant studies on \emph{Sueca} have been found, the research includes algorithms used in similar card games. It also presents existing robots that will allow the analyses on human-robot interaction.


\subsection{Card games programs}
\todo{change the title}
 
There are many games that AI have been solving over the years. Although, the definition of "games" usually refers to zero-sum and perfect information ones. This kind of games is commonly solved by creating a tree search representing all the possible states. The greatest achievements are generally related to finding good heuristics to refine the search and also good prunings to reduce the space of search. An example of a game with the mentioned characteristics is chess.
\todo{Do I refer the Russel Norvig book?!}

As described in section \ref{background}, \emph{Sueca} is considered an imperfect information game. This uncertainty produces a high branching factor in the tree search. Thus, problems of this nature are being solved by other methods. Finding the optimal solution along the state tree is fairly impractical. Some possible approaches include considering the belief states about other players or trying to generalize the reward of a move by sampling the game considerable times.


\subsubsection{Monte Carlo Tree Search}




\subsection{Human-robot interaction}
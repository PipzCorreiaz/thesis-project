\section{Evaluation Methodology} \label{sec:evaluation}

The correctness and benefits of the proposed work must be evaluated.
Firstly, a proper user study of elderly playing \emph{Sueca} with \gls{emys} will be settled.
Each group will play a tournament with two different versions of the embodied agent:
\begin{itemize}
\item An agent that plays the game with few or nonexistent social behaviours;
\item An agent that plays the game and reacts according to the game state with verbal and nonverbal cues.
\end{itemize}
After the tournament, each person will answer a questionnaire in order to evaluate the individual experience and feelings during the game.
On one hand, the questionnaire will compare the experience with a robot versus the traditional card game.
On the other hand, it will compare the two versions of the agent and understand how the presence of each version of the robot was perceived.

Secondly, finding a group of elderly to do a user study is not so simple.
Considering such constraints, the evaluation of the artificial player will not be made with aged people.
Due to the college environment where this work takes place, the university community will attend to tournaments in order to do this evaluation.
The performance of the artificial player will be measured by the percentage of winning games.
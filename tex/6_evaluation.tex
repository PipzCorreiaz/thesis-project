\section{Evaluation Methodology} \label{sec:evaluation}

The correctness and benefits of the proposed work must be evaluated.
Firstly, it will be settled a proper user study of elderly playing \emph{Sueca} with \gls{emys}.
Each group will play a tournment with two different versions of the embodied agent:
\begin{itemize}
\item An agent that plays the game with few or nonexisting social behaviours;
\item An agent that plays the game and reacts according to the game state with verbal and nonverbal cues.
\end{itemize}
After the tournment, each person will answer a questionaire in order to evaluate the individual experience and feelings during the game.
On one hand, the questionaire will compare the experience with a robot versus the tradicional card game.
On the other hand, it will compare the two versions of the agent and understand how the presence of each version of the robot was perceived.

Secondly, finding a group of elderly to do a user study is not so simple.
Considering such problems, the evaluation of the artificial player will not be made with aged people.
Due to the college environment where this work takes place, the university community will attend to tournments in order to do this evaluation.
The performance of the the artificial player will be measured by the percentage of winning games.
\section{Evaluation Methodology} \label{sec:evaluation}

The current section aims to explain the methodology that will be used in order to evaluate the correctness and benefits of the proposed work.

In order to evaluate the proposed artificial player that will be developed using the \gls{pimc} algorithm, there are two main considerations: parametrization and performance.
The parametrization issue will be addressed by measuring the time of both the offline pre-computation and runtime decision, varying the value of the sampling limit parameter.
The average points per tournament will be used as a performance measure, and it will be compared to naive approaches (e.g. rule-based).
In addition, our artificial player will play against humans in order to evaluate its performance.
The last mentioned evaluation will not be developed with aged people, considering that finding a group of elderly to do it is not simple, and it will instead use the university community.

%Finding a group of elderly to do it is not simple, and considering such constraint, the evaluation of the artificial player will not be made with aged people.
%Due to the college environment where this work takes place, the university community will attend to tournaments in order to do this evaluation.


Concerning the integration of a social embodied agent into the game scenario, a proper user study of elderly playing \emph{Sueca} with \gls{emys} will be settled.
Each group will play a tournament with two different conditions of the embodied agent:
\begin{itemize}
\item An agent that plays the game with few or nonexistent social behaviours;
\item An agent that plays the game and reacts according to the game state with verbal and nonverbal cues.
\end{itemize}
After the tournament, each person will answer a questionnaire in order to evaluate the individual experience.
This questionnaire aims to measure the participants' perception of the robot and also their presence perception of the embodied agent, using, respectively, the Godspeed questionnaire \cite{Bartneck2008} and Networked Minds \cite{Biocca2001}.

%On one hand, the questionnaire will compare the experience with a robot versus the traditional card game.
%On the other hand, it will compare the two versions of the agent and understand how the presence of each version of the robot was perceived.


%Secondly, to evaluate \emph{Sueca} artificial player, another user study must be settled.
%Finding a group of elderly to do it is not simple, and considering such constraint, the evaluation of the artificial player will not be made with aged people.
%Due to the college environment where this work takes place, the university community will attend to tournaments in order to do this evaluation.
%The performance of the artificial player will be measured by the percentage of winning games.
\subsection{Game theory concepts}

Game theory studies decision making problems involving multiple decision makers.
A problem of this nature is usually called a game and defines a set of constraints to the players' actions.
It also studies the strategies these players might take and the properties of each game.
%This subject is applied to many different areas, such as economics, political science, biology, or computer science.

Each decision maker tries to maximise the payoff/reward of his possible actions and one possible approach to do that is to consider the opponents' actions.
Consequently, the Nash-equilibrium \cite{Nash1950} of a game is a stable strategy for every player and occurs when each player chooses the best strategy for himself, considering their opponents have the same behaviour.
Moreover, each player cannot have a better benefit by changing his strategy.

\begin{figure}
\centering
\includegraphics[width=1\textwidth]{./img/gamesHierarchy}
\caption{Hierarchy of games}
\label{fig:games}
\end{figure}

Figure~\ref{fig:games} shows how games can be hierarchically categorised \cite{Osborne2011}.
In a cooperative game, players cooperate with one another in order to achieve a common goal.
Alternatively, in a non-cooperative game, players work independently for their own purposes.
Non-cooperative games can also be branched in two forms: normal and extensive form \cite{Shoham2010}.
A normal form game can be defined as the tuple $(N,(A_k)_{k=1}^{N},(u_k)_{k=1}^{N})$, where:

\begin{itemize}
\item $N$ is the number of players;
\item $A_k$ is the finite set of available actions for the $k$-th player;
\item $u_k$ is the payoff for the $k$-th player.
\end{itemize}

Additionally, considering the players' payoffs, another relevant concept is the zero-sum game, where the sum of all players' payoffs is zero.
For instance, in a zero-sum 2 players game, $u_1 = -u_2$.
Although the normal form games assume that players' actions are made simultaneously, in the extensive form games, the players' actions are sequential.
This evidence leads to another branching in the hierarchy of games and, consequently, an extensive game can be considered as a perfect information and an imperfect information game.
The first means each player knows exactly the real state of his opponents, (e.g. Chess).
The second means opponents' state is hidden (e.g. card games).

Considering the focus of this work on a card game, another relevant point on top of hidden information games is the definition of information set.
An information set aggregates several nodes representing unknown information such as choice nodes. In a card game, for instance, the information set of a player, whose cards are hidden, corresponds to all the possible combinations of his hand.




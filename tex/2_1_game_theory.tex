\subsection{Game theory concepts}

Game theory studies decision making problems involving multiple decision makers.
Each one of the decision makers tries to maximaize the payoff/reward of his possible actions.
A problem of this nature is usually called game and defines a set of constraints to the players and their actions.
Game theory also studies the strategies these players might take and also the properties of each game.
Nowadays this subject is applied to many different areas, such as economics, political science, biology or computer science.\\

\begin{figure}
\centering
%\includegraphics[width=0.5\textwidth]{./img/focusGroup}
\caption{Hierarchy of games}
\label{fig:games}
\end{figure}

Games can be categorized according to the hierarchy of Figure~\ref{fig:games}.
In a cooperative game, players cooperate with one another in order to achieve a common goal.
Alternatively, in a non-cooperative game, players work independently for their own purposes.
Non-cooperative games can also be branched in two forms: normal form and extensive form \cite{Shoham2010}.
A normal form game can be defined as the tuple $(N,(A_k)_{k=1}^{N},(u_k)_{k=1}^{N})$, where:
\begin{itemize}
\item $N$ is the number of players;
\item $A_k$ is the finite set of available actions for the $k$-th player;
\item $u_k$ is the payoff for the $k$-th player.
\end{itemize}
Considering the players' payoffs, there is another relevant concept, the zero-sum game.
In a zero-sum game, the sum of all players' payoffs is zero.
For instance, in a zero-sum 2 players game, $u_1 = -u_2$.
Although a normal form game assumes that players' actions are made simultaneously, in an extensive form game the players' actions are sequential.
This evidence leads to another branching in the hierarchy of games.
On one hand, a game can be considered a perfect information.
This means each player knows exactly the real state of his opponents (e.g. Chess).
On the other hand, in an imperfect information game, the state of opponents is hidden (e.g. card games).


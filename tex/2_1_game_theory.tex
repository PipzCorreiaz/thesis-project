\subsection{Game theory concepts}

Game theory studies the decision making problems.
These problems are called games and each one of them defines a set of constraints to the players and their actions.
Game theory studies the strategies these players might take and also the properties of each game.
Nowadays this subject is applied to many different areas, such as economics, political science, biology or computer science.

The basic concept in game theory is player.
A player is an entity that makes decisions.
Sometimes the actions of more than one player can be grouped for some purpose in the game, which defines a cooperative game.
Another two relevant concepts are strategic and extensive games.
In a strategic game, every player defines a strategy and makes a decision simultaneously.
Otherwise it is an extensive game.
This means each player acts with a certain order and his strategies are defined before decision points.
Extensive games can also be considered as perfect or imperfect information.
In a perfect information game, every player knows the real state of other players (e.g. Chess).
Conversely, a game with imperfect information implies that a player's state is hidden from the others (e.g. Bridge).

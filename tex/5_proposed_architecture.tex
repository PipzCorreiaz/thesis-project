\section{Proposed Architecture} \label{architecture}

The current section will describe how the two first goals of
Section~\ref{sec:goals} will be addressed.
Firstly, it presents the chosen approaches in order to build the \emph{Sueca} player (Section \ref{sec:sueca_solution}).
Lastly, it introduces the conceptual model and the architecture of the physical embodied agent that reacts socially according to the game state (Section \ref{sec:social_solution}).


\subsection{\emph{Sueca}}
\label{sec:sueca_solution}

Currently, there are not artificial players of the \emph{Sueca} card game.
Research has shown that the state of the art of imperfect information games are based on Monte Carlo methods.
To build the \emph{Sueca} card game, the chosen approach is similar to what Furtak et. al. have done in the Skat card game.
These two games are identical, excluding the nonexistent bidding phase on \emph{Sueca}.
Furtak et. al. have explored how \gls{pimc}'s results vary according to some of the game's properties and proved its benefits on Skat.
Due to the affinity between the two games, it is predictable that the results of applying \gls{pimc} to \emph{Sueca} are also satisfying.

Furthermore, this \emph{Sueca} player will play against aged people.
Since they are not world champions or are not even at a professional level, the power of the artificial player must be balanced.
On one hand, the idea is to create a challenging environment for the elderly.
On the other hand, an existing concern is not to devastate the self-esteem of them.
The motivation of this work is to create a pleasing and, at the same time, stimulating activity for the elderly.

Additionally, a possible development strategy to follow is \gls{iimc}.
Furtak et. al. shown how this algorithm can improve the results of their Skat player.
Depending on the results of applying \gls{pimc} to our domain, it will be considered a further upgrade to the \emph{Sueca} player.

To reinforce the sampling phase of \gls{pimc}, it will be used an opponent modelling.
Instead of using the random sampling method of the original \gls{mcts}, this opponent modelling...
\todo{Conclude the opponent modelling propose.}


\subsection{The social robot in the game context}
\label{sec:social_solution}


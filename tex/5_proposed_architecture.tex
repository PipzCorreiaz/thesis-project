\section{Proposed Architecture} \label{sec:architecture}

The current section describes how to address the development of an artificial \emph{Sueca} player and its integration into an embodied agent that interacts with other players during the game.
First of all, it presents the chosen approaches in order to build the artificial \emph{Sueca} player (Section \ref{sec:sueca_solution}).
Lastly, it introduces the conceptual model and the architecture of the physical embodied agent that reacts socially according to the game state (Section \ref{sec:social_solution}).


\subsection{\emph{Sueca}}
\label{sec:sueca_solution}

Currently, there are no artificial players of the \emph{Sueca} card game and the review of the related work was focus on other hidden information games.
Research has shown that the state of the art of imperfect information games is based on Monte-Carlo Methods.
To build the \emph{Sueca} card game, the chosen approach is similar to what Buro et al. have done in the Skat card game, since these two games are identical, excluding the nonexistent bidding phase on \emph{Sueca}.
Moreover, Furtak et al. have explored how \gls{pimc}'s results vary according to some of the game properties and have proved its benefits on Skat \cite{Long2010}, and, due to the affinity between the two games, it is predictable that the results of applying \gls{pimc} to \emph{Sueca} are also satisfactory.


The idea is to evaluate \gls{pimc} responses in our domain, since no previous studies have been done.
However, if further improvements are needed, \gls{ismcts} and \gls{iimc} are still available options.
These two approaches were not our first choice due to the computational burden they introduce.


Furthermore, this \emph{Sueca} player will play against aged people.
Since they are not world champions or are not even at a professional level, the power of the artificial player must be balanced.
On one hand, the idea is to create a challenging environment for the elderly.
On the other hand, an existing concern is not to devastate their self-esteem.
The motivation of this work is to create a pleasing and, at the same time, stimulating activity for the elderly.
%Additionally, a possible development strategy to follow is \gls{iimc}.
%Furtak et. al. shown how this algorithm can improve the results of their Skat player \cite{Furtak}.
%Depending on the results of applying \gls{pimc} to our domain, it will be considered a further upgrade to the \emph{Sueca} player.
As a result, to reinforce the sampling phase of \gls{pimc}, an opponent model similar to what has been done in Poker will be used \cite{Ponsen2008}.
With this technique, we aim to approximate our artificial agent to a common \emph{Sueca} player.


This model will include cards and actions predictions.
Instead of using the random sampling method of the original \gls{pimc}, cards probabilities will influence the cards sampling in each iteration of the algorithm.
After sampling each world, actions predictions will be used in the simulation of a game to influence the opponents' moves.
This technique also aims to introduce some of the common mistakes that the usual player makes, instead of always considering optimal moves.


In order to model opponents, several instances of \emph{Sueca} games will be collected.
To easily register this game playing logs, an additional platform must be created.




\subsection{The social robot in the game context}
\label{sec:social_solution}

Along with the \emph{Sueca} player, this work aims to develop a robot that is socially present in the environment of the game scenario.
In order to achieve this goal, many concerns arise.
T\textit{}he model presented in Figure~\ref{fig:model} tries to solve and organise all the components involved.

\begin{figure}[ht]
  \centering
    \includegraphics[width=1\textwidth]{./img/model}
  \caption{Structure of the social robot that plays \emph{Sueca}}
\label{fig:model}
\end{figure}

This model distinguishes physical components from virtual ones.
Some entities are not detailed on the scope of this project and are presented as both physical and virtual components.
The human players, Users, play with physical cards on top of a Touch Table, and their game actions are managed by Game Application and communicated to both the \gls{ai} module and the Decision Maker.
The Perception Manager receives information from all the Sensory Components.
The \gls{ai} module includes all the reasoning about the game and decides the robot's next move.
However, the embodied agent's actions also involve social behaviours, and Decision Maker is the responsible module for this management.
The Decision Maker balances the \gls{ai} move and game information, depending on the situation, in order to produce an appropriate sequence of behaviours and inform them to Behaviour Planner.
For instance, being the last player of a trick and taking the two highest cards of a suit, by playing a trump card, is an exciting move and should produce an equally exciting reaction on the robot.
Lastly, the Behaviour Planner, after receiving high-level intention-directed instructions, builds a suitable plan to execute the chosen instructions, considering the state of the embodied agent, information from Perception Manager, and additional game information from the Game Application.

The architecture that will instantiate the model described above is partially decided.
The virtual layer will be mainly covered by Thalamus framework, which enables the communication between the mentioned entities \cite{Ribeiro}.
The chosen Behaviour Planner is Skene \cite{Ribeiroa}.
The \gls{ai} module will be processed offline with the algorithms described in Section~\ref{sec:sueca_solution}.
The embodied agent will initially be \gls{emys} due to its expressiveness, although other robots will be considered, depending on the users' preferences.
Finally, the undecided components are the sensory inputs.
Since Pereira et al. have shown the importance of collecting data from user studies, these components will be settled further, after some field research with \emph{Sueca} players.










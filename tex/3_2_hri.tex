\subsection{Human-robot interaction}

Regarding the goals of this project, it is crucial to investigate and evaluate the state of the art of \gls{hri}.
The social robot that is proposed aims to interact with aged people in a card game scenario.
Due to this ambition, there are two clear branches that must be studied.
Firstly, the existing robots with an elderly care purpose.
Secondly, how social agents have been integrated into games.
The next subsections will address these points.



\subsubsection{Robots in elderly care}


The greying of population is an undeniable demographic question.
As a result, assisting the elderly in their daily living is a worrying subject.
In order to address this problem, robots can be a valuable aid.
Replacing human carers is an ambitious task due to the limited capabilities of current robot technology \cite{Walters2013}.
In spite of these limitations, robots can be useful in more distinct tasks.

In 2009, Broekens et. al. have analysed and reviewed the most relevant literature about social robots in elderly care until that date \cite{Broekens2009}.
The authors have categorised assistive robots for elderly as shown in figure~\ref{fig:categorization}.
The first division distinguishes social robots from nonsocial robots.
These nonsocial ones are used for rehabilitation purposes and physical assistance, such as a smart wheelchair or an artificial limb.
Regarding the main purposes of this work, nonsocial robots will not be discussed.
The social ones should be perceived as social entities due to their interaction with humans.
They can also be divided into two different sets, service type and companion type.
The intersection of these two sets represents some of the robots that are used in both ways and cannot be strictly categorised.

\begin{figure}[h!]
  \centering
    \includegraphics[width=0.7\textwidth]{./img/categorization_robots}
  \caption{Categorization of assistive robots for elderly}
\label{fig:categorization}
\end{figure}

A well known social service robot is \emph{Pearl}.
It was developed in the Carnegie Mellon University within the Nursebot Project \cite{Pollack2002}.
\emph{Pearl} can be defined as a nursebot considering its main goal is guiding aged people.
One of this autonomous robot's duty is guiding the elderly through their environment.
Likewise, it frequently reminds them about their daily activities, such as eating or taking the medicines.
In other words, this functional assistant is capable of giving advice and providing cognitive support.
When analysing \emph{Pearl} through a more general \gls{ai} point of view, this robot is equipped with many different technologies.
Firstly, it has a speech recognition module and also has speech synthesis.
Secondly, it has stereo camera systems and performs a fast image processing including face recognition.
Lastly, \emph{Pearl} also provides a navigation system and its body is touch sensitive.

Another two similar robots of service type are \emph{RoboCare} \cite{Bahadori} and \emph{Care-O-bot II} \cite{Graf2004}.
They both are autonomous and provide indoor guidance to the elderly.
Their advanced domotic components together with strong planning and scheduling frameworks can improve the independence of their owners.
The aid these service type robots grant to elderly represent most of their basic daily activities.
Consequently, the involved concerns are amplified when comparing to the proposed robot that plays a card game.
These worries are reflected, for instance, in the extensive amount of sensors service robots should include.

\emph{Paro} is considered a companion robot.
This seal shaped robot is used as medical therapy for elderly.
Since 2003 the work of Wada et. al. tries to evaluate psychological, physiological and social effects of \emph{Paro} on the residents of care house \cite{Wada2003}\cite{Wada2005}\cite{Wada2007}.
This robot contains a behaviour generation system that provides proactive, reactive and physiological reactions, such as poses or motions, looking at the direction of a sound and sleeping, respectively.
Their studies of both three weeks and one year have shown improvements in residents' moods, depression and stress levels.
They also reveal an increase of their social interactions.
The goal of such a robot is fully inspired in animal-assisted treatments, which have studied benefits in humans' health.
However, hospitals and health centres do not allow animals due to hygienic and safety reasons.
Hence, researchers found a great opportunity to build similar robotic animals.

Another example of a purely companion robot is the \emph{Huggable} \cite{Stiehl2005}.
It is covered of extremely sensitive touch sensors.
The \emph{Huggable} not only detects hard and soft touches, but also distinguishes between an object and a human touch.
The teddy bear shaped robot is connected to a computer in the nurses' station and allow the staff to access the sensory input data.
A possible usage of that data in an hospital is detecting fear or insecurity by the way people hold the robot.
Purely companion robots in elderly care have only been applied to people with some kind of psychological or physiological disorder.
As a result, these studies have distinct target audiences and also different concerns when compared to the purposes of the proposed robot.


\emph{Aibo} illustrates a robot that can be assigned to both the service type and the companion type \cite{Fujita1983}.
It is considered by its creators as an entertainment robot.
This is a dog shaped robot and its appearance tries to maintain a lifelike experience to its owners.
The authors affirm that a fault in a service robot might cause serious problems.
This idea influenced them to create a robot that would not be life-threatening, such as \emph{Aibo}.
Other researchers started to study the acceptance and effects of this robot on elders with severe dementia.
Their study revealed a relevant increase of social actions, emotions and feelings of comfort about past memories.
Considering the goals of the proposed robot, \emph{Aibo} is much more close to our goals than the other mentioned robots.
A possible explanation is that both robots have an entertainment purpose.



\subsubsection{Social robots in games}

The idea of entertainment robots, that has been mentioned previously, is expanding and becoming more frequent.
The general goal is to create a social robot to interact with humans, however it is done using a specific entertainment activity.
These activities should be lifelike experiences and provide pleasure and enjoyment feelings.
Depending on the target audience, they can also be included in more challenging or even pedagogic activities.

Iolanda et. al. uses the \emph{iCat} robot in a chess game scenario with children \cite{Leitea}\cite{Castellano2010}\cite{Leite}.
This chess companion also has the role of a tutor due to the help it provides during the game.
The \emph{iCat} expresses opinions about children' moves so that they can improve their chess skills.
After their first pilot studies, the authors revealed the need of including social and cognitive abilities.
This idea is commonly referred as empathy, which involves for example recognising users' expressions or considering others' affective states.
For instance, when a child is losing, the \emph{iCat} comments about his moves should not cause embarrassment.
As a result, the authors' further studies introduced into the \emph{iCat} affect recognition in order to improve the robot's social cues.

The \emph{iCat} chess player has some similarities and differences with the proposed robot of this work.
On one hand, including empathic behaviour to robots usually leads to more engaging, natural and likable experiences to users.
On the other hand, the \emph{iCat} in this scenario needs access to more details of users' emotional state because of its tutoring advices. Our \emph{Sueca} player will not advise other players about their actions, instead it will comment the game state.
Additionally, the target audience is clearly different and may lead to different concerns.


Another example of a robot integrated into a game scenario is the Risk player by Pereira et. al \cite{Pereira}.
The goal of their work was to create a robot that interacts with humans and is perceived as socially present in long-term interactions.
The authors presented some guidelines in order to improve social presence and how they implemented them in the \emph{EMYS} robot for the mentioned scenario.
Firstly, a physical embodiment can provide interactivity and therefore causes the belief of social presence and improves face-to-face interactions.
Thus, \emph{EMYS} produces non-verbal interactions through a gazing system and a speech direction detector.
In addition, it is capable of giving verbal feedback using a topology of speeches according to the game state.
Moreover, they include an emotion or appraisal system that considers the values of some variables to improve the robot's behaviours.
For instance, every event is rated with a relevance value and the robot only comments important moves.
Another example is measuring the power of each player.
Since Risk is about conquering and controlling, this power measure is used to shape the robot's mood and defining its strategy to play.
Equally important are the simulation of social roles and the luck perception when rolling the dice.
Considering their goals were also focused on long term interactions, this Risk player recognises faces and greet people mentioning past events.
All the described behaviours were inspired by user studies.

Pereira's work is far the most similar to the purposes of our goals.
They demonstrate how to enrich the Risk game experience with a robot capable of social behaviours at a human level.
The main difference from the proposed \emph{Sueca} player is the game.
Since no relevant user studies have been done with \emph{Sueca}, applying the Risk' constraints to the \emph{Sueca}'s scenario would lead to inconsistencies.
However, an analogous approach might be taken.


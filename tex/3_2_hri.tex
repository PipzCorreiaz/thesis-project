\subsection{Human-robot interaction}

Regarding the goals of this project, it is crucial to investigate and evaluate the state of the art of \gls{hri}.
The social agent that is proposed aims to interact with aged people in a card game scenario.
Due to this ambition, there are two clear branches that must be studied.
Firstly, the existing robots with an elderly care purpose.
Secondly, how social agents have been integrated into games.
The next subsections will address these points.



\subsubsection{Robots in elderly care}


The greying of population is an undeniable demographic question.
As a result, assisting the elderly in their daily living is a worrying subject.
In order to address this problem, robots can be a valuable aid.
Replacing human carers is an ambitious task due to the limited capabilities of current robot technology.
\todo{Cite Dautenhahn}
Although these limitations, robots can be useful in more distinct tasks.

In 2009, Broekens have analysed and reviewed the most relevant literature about social robots in elderly care until that date.
The authors have categorised assistive robots for elderly as shown in figure~\ref{fig:categorization}.
The first division distinguishes social robots from nonsocial robots.
These nonsocial ones are used for rehabilitation purposes and physical assistance, such as a smart wheelchair or an artificial limb.
The social ones should be perceived as social entities due to their interaction with humans.
They can also be divided into two different sets, service type and companion type.
The intersection of these two sets represents some of the robots that cannot be strictly categorized.

\begin{figure}[h!]
  \centering
    \includegraphics[width=0.7\textwidth]{./img/categorization_robots}
  \caption{Categorization of assistive robots for elderly}
\label{fig:categorization}
\end{figure}


\subsubsection{Social robots in games}
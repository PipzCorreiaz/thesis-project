\section{Introduction} \label{introduction}

The ageing population has been increasing over the years.
As a result, some concerns about the elderly have also augmented.
Most of the time, there are special needs due to their physical disabilities.
Similarly, there is a worry about keeping them entertained.
It is crucial to find appropriate activities for them.
These activities may range from training their cognitive functions to just accompanying them.
Hence, all these concerns are commonly being solved with the help of technology.
For instance, there is software suitable for training memory problems due to Aphasia \cite{Pompili2011}.
Another example that illustrates the presence of technology in the elderly lives is a robot.
There already are robots in their homes reminding them to take the medicines or vacuuming the floor.
However, occupying their free time with pleasuring activities continues to be a challenging task.


In parallel with this idea, computer programs that play games have been an interesting challenge over the years for \gls{ai}.
From board games to card games, or even role-playing games, the idea is to create rational agents capable of evaluating the game and achieving the best outcome.
Deep Blue, Chinook and Watson are good examples that have raised the bar for developing this kind of agents.
Deep Blue is a remarkable chess player and has defeated the human world champion in 1997 \cite{Campbell2002}.
Schaeffer et al. have solved Checkers with Chinook program and proved the game leads to a draw with two optimal players \cite{Schaeffer1996}.
Lastly, Watson is the \gls{qa} system that has beat the two highest ranked Jeopardy players in 2011 \cite{Ferrucci2010}.
\gls{qa} systems are so impressive due to the scope of \gls{ai} they include (i.e. natural language processing, machine learning, knowledge representation, automated reasoning, and information retrieval).

Besides building programs that try to think rationally or humanly, \gls{ai} has also another branch that aims to act humanly \cite{Russell2009}.
This concern arises from the inclusion of robots in humans' life.
Most of these robots' purpose is doing certain tasks that are currently being done by humans (i.e. vacuum the floor, play football or drive a car).
Consequently, they have to behave properly in those environments.
Considering these robots are surrounded by humans, it is a relevant concern with the way they interact and communicate with people.
The human-robot interaction field explores the social integration of robots with humans.

Card games are a good example of an activity that aged people enjoy doing to occupy their free time.
This opportunity is also an unsolved challenge for \gls{ai}.
Furthermore, including this artificial player into an embodied agent brings \gls{hri} concerns.
The goal of this project is to integrate a social robot with aged humans in a card game scenario.
The chosen card game is \emph{Sueca}, a well known Portuguese game among the ageing population.
It is a great opportunity to relate the concerns mentioned above.
On one hand, this agent should play the game correctly.
After analysing the given hand, dealt at the beginning of the game, it should make good choices about what cards should be played.
On the other hand, this embodied agent should act accordingly to the environment of this scenario.
Since \emph{Sueca} is a four-player game with two teams, it involves an opponent and a companion role.
Every player should be engaged in the game with its verbal interaction and behaviours.


\subsection{Goals}
\label{sec:goals}

The main goals of this project are:
\begin{itemize}
\item Develop an agent capable of playing and winning the \emph{Sueca} card game;
\item Include social behaviours on an embodied agent in order to act according to the game state;
\item Evaluate the correctness and advantages of the proposed system.
\end{itemize}

\subsection*{\centering*}

The next section presents some background research that helps the reader understanding the problems further mentioned (Section~\ref{sec:background}).
The report proceeds with the state-of-art of playing card-games and human-robot-interaction with the elderly (Section~\ref{sec:related-work}).
Additionally, it reveals a pilot user study that has already been done (Section~\ref{sec:user-studies}).
Finally, it presents the proposed architecture, its evaluation methodology and the final conclusions (respectively, Sections~\ref{sec:architecture}, \ref{sec:evaluation} and \ref{sec:conclusion}).



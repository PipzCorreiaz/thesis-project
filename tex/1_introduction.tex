\section{Introduction} \label{introduction}

The ageing population has been increasing over the years and some concerns about the elderly have also augmented.
Most of the time, they have special needs due to their physical or mental disorders.
Similarly, there is a worry about keeping them occupied and entertained, and so, it is crucial to find appropriate activities for them.
Some of these concerns are recently being solved with the help of technology that may range from computer programs to intelligent robots.
For instance, elderly with mobility disabilities may be guided with a robot while walking through the house \cite{Pollack2002}.
Another example to train and stimulate a brain damaged illness is a suitable software for training memory problems of Aphasia \cite{Pompili2011}.


However, occupying the elderly's free time with pleasuring activities continues to be a challenging task, specially when dealing with aged people with no serious health problems that are still capable of doing many activities they did before.
Finding appropriate activities supported by technology may include the training of their cognitive functions or just accompany them.
For instance, companion robots and virtual agents are already present in care homes \cite{Wada2003,Wada2005,Wada2007,Zancanaro2013}.
However, joining theses two purposes of accompany them and training their reasoning might result, for example, in an artificial game player.
A card game is, for instance, a good example of an activity that aged people enjoy doing to occupy their free time.
To illustrate this suggestion, existing embodied agents are the iCat chess tutor \cite{Affective2007} and the \gls{emys} Risk player \cite{Pereira}, nevertheless these scenarios were not applied to aged people.


In parallel with the elderly concerns and their technological answers, computer programs that play games have been an interesting challenge over the years for \gls{ai}.
From board games to card games, or even role-playing games, the goal is to create rational agents capable of evaluating the game and achieving the best outcome.
Deep Blue, Chinook and Watson are good examples that have raised the bar for developing this kind of agents.
Deep Blue is a remarkable chess player and has defeated the human world champion in 1997 \cite{Campbell2002}.
Schaeffer et al. have solved Checkers with Chinook program and proved the game leads to a draw with two optimal players \cite{Schaeffer1996}.
Lastly, Watson is the \gls{qa} system that has beat the two highest ranked Jeopardy players in 2011 \cite{Ferrucci2010}.
\gls{qa} systems are so impressive due to the scope of \gls{ai} they include (i.e. natural language processing, machine learning, knowledge representation, automated reasoning, and information retrieval).
All these agents are good baselines to improve \gls{ai} in games.


Besides building programs that try to think rationally or humanly, \gls{ai} has also another branch that aims to act humanly \cite{Russell2009}.
This concern arises from the inclusion of robots in humans' life.
%Most of these robots' purpose is doing certain tasks that are currently being done by humans (i.e. vacuum the floor, play football or drive a car).
Consequently, they have to behave properly in those environments.
Considering these robots are surrounded by humans, the way they interact and communicate with people is a relevant concern.
The human-robot interaction field explores the social integration of robots with humans.


Consequently, technology that covers this point has concerns related with the feeling of objectification, loss of responsibilities and privacy \cite{Should2010}.


Card games are a good example of activities that aged people enjoy doing to occupy their free time.
Moreover, some of existing card games still are unsolved challenges for \gls{ai}.
Additionally, including this artificial player into an embodied agent brings \gls{hri} concerns.
As a result, the goal of this project is to integrate a social robot with aged humans in a card game scenario.
The chosen card game is \emph{Sueca}, a well known Portuguese game among the ageing population.
It is a great opportunity to relate the concerns mentioned above.
On one hand, this agent should play the game correctly.
After analysing the given hand, dealt at the beginning of the game, it should make good choices about what cards should be played.
On the other hand, this embodied agent should act accordingly to the environment of this scenario.
Since \emph{Sueca} is a four-player game with two teams, it involves an opponent and a companion role.
\todo{Desenvolver esta questao}
Every player should be engaged in the game with its verbal interaction and behaviours.



\subsection{Goals}
\label{sec:goals}

The main goals of this project are:
\begin{itemize}
\item Develop a robotic agent capable of playing and winning the \emph{Sueca} card game;
\item Include social behaviours on an embodied agent in order to act according to the game state;
\item Evaluate the correctness and advantages of the proposed system.
\end{itemize}

\subsection*{\centering*}

The next section presents some background research that helps the reader to understand the problems further mentioned (Section~\ref{sec:background}).
The report proceeds with the state-of-art of playing card-games and human-robot-interaction with the elderly (Section~\ref{sec:related-work}).
Additionally, a pilot user study is revealed (Section~\ref{sec:user-studies}).
Finally, proposed architecture is presented, its evaluation methodology and the final conclusions (respectively, Sections~\ref{sec:architecture}, \ref{sec:evaluation} and \ref{sec:conclusion}).



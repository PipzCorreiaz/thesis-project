\section{Introduction} \label{introduction}

Computer programs that play games have been an interesting challenge over the years for \gls{ai}. From board games to card games, or even role-playing games, the idea is to create a rational agent capable of evaluating the game and trying to achieve the best outcome. Deep Blue has raised the bar for developing this kind of games, considering it was the first one capable of defeating a human world champion in 1997.

Besides building programs that try to think rationally or humanly, AI has also another branch that aims to act humanly. This concern arises from the inclusion of robots into man's life. Most of theses robots' purpose is doing certain tasks that are currently being done by humans. Consequently, they have to behave properly in that environment. Considering these robots are surrounded by humans, there is also another relevant concern with the way they interact and communicate with people. The human-robot interaction field explores the social integration of robots with humans.

In parallel with these ideas, another worry of our society relies on the ageing population. Most of the time, the elderly have special needs due to their disabilities. Similarly, there is the need to keep them entertained and also train their cognitive function. 

As a result, the goal of this project is to integrate a social robot with three elderly humans in a card game scenario. The chosen card game is \emph{Sueca}, a well known Portuguese game among the ageing population. It is a great opportunity to relate the concerns mentioned above.

On one hand, this agent should play the game correctly. After analyzing the given hand, dealt at the begging of the game, it should make good choices about what cards should it play.
\todo{Not sure about the verb tense with should}

On the other hand, this embodied agent should act accordingly to the environment of this scenario. Since \emph{Sueca} is a four-player game with two teams, it involves an opponent and companion role. Every player should be engaged in the game with its verbal interaction and behaviors.

\todo[color=yellow]{Talk about the proposed architecture?}

The next section presents the state-of-art of playing card-games and human-robot-interaction with elderly. The report proceeds with the proposed architecture (section \ref{architecture}). Finally, it covers the evaluation methodology (section \ref{evaluation}). 
\section{Introduction}
Playing games have been an interesting challenge over the years for artificial intelligence. On one hand, there is the notion of rational agent capable of evaluating the game and trying to achieve the best outcome. On the other hand, acting humanly is another concern when applying this agent to an embodied robot.

Nowadays, another concern of our society relies on the ageing population. Due to their needs, some robots have emerged to help them with daily duties or to be their companions.

On the top of these two main ideas, arises the great opportunity to solve \emph{Sueca}. Besides being unexplored by an artificial intelligence point of view, it is a well known game among the elderly in Portugal.

\emph{Sueca} is a trick-taking card game played with a standard fourty-card deck. This Portuguese variation of the Italian game \emph{Briscola} differs from the original on the number of players and the distribution of cards. The deck if shared among the four players and these are divided into two teams. Hence the game is considered as both competitive and cooperative.

The next section covers the goals of this project. Then, Section 3 presents the state-of-art of playing card-games and human-robot-interaction with elderly. The report proceeds with the proposed architecture (Section 4) and, finally, its evaluation methodology (Section 5). 
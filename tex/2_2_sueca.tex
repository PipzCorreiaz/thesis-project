\subsection{The game of \emph{Sueca}}

\emph{Sueca} is a card game belonging to the class of trick-taking.
This means the game has a finite number of rounds, called tricks.
In this case, there are ten tricks, since the deck has forty cards equally distributed among the four players.
This game uses the standard French card deck, excluding the rank 8 through 10.
Most of trick-taking card games count the number of winning tricks to determine the winner.
However, \emph{Sueca} assigns points to the cards, according to the Table~\ref{tab:points-table}.
All valued cards sum 120 points, which means a team with more than 60 points wins the game.
Each player is paired with the player in front of him, and the two adjacent players form the opposing team.
Hence the game is considered as both competitive and cooperative.

\begin{table}[ht]
\centering
\begin{tabular}{|c|c|c|c|c|c|c|}
\hline
Cards  & 2-6 & Q & J & K & 7  & A  \\ \hline
Points & 0   & 2 & 3 & 4 & 10 & 11 \\ \hline
\end{tabular}
\caption{Rank of cards per suit and respective reward values}
\label{tab:points-table}
\end{table}

After the deck has been shuffled and divided, the dealer chooses the top or bottom card to be the trump suit and distributes the cards among all players.
The remaining rules are quite similar to other trick-taking games.
As mentioned in previous subsection, \emph{Sueca} is a nondeterministic game, since it includes what is called the element of chance by the cards being dealt randomly at the beginning.
Additionally, since the cards of each player are hidden from the other players, this is considered as an imperfect information game.
There are almost $1.9\times10^{22}$ possible card distributions\footnote{$4\times\Comb{40}{10}\times\Comb{30}{10}\times\Comb{20}{10}$}.
%Nevertheless, considering the initial hand of the agent, the uncertainty about other players' hands decreases to $5.9\times10^{12}$ \footnote{$\Comb{30}{10}\times\Comb{20}{10}$}.
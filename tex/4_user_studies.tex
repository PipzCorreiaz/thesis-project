\section{User Centered Studies} \label{sec:userstudies}

Developing this kind of technology, such as a robot, for aged people brings some delicate questions.
The potential users sometimes have few or nonexistent experience with technology.
Regarding this evidence, it is difficult for them to understand how robots work and what can they actually do.
This problem arises specially from the need of doing user studies.
The most commonly used technique to present ideas or to receive user feedback is \gls{vhri}.

Kerstin Dautenhahn  et. al. introduced in 2013 the \gls{thri}, that aims to integrate theatre into user studies.
Since \gls{vhri} does not provide live experiences, this new methodology is considered more immersive.
Their last \gls{thri} study has been done in an elderly care home and its purpose was analysing both residents and carers' opinions.
The results and conclusions of the mentioned user study revealed a problem with questionnaires.
Due to the ageing-associated disorders, such as poor eyesight or arthritis in hands, part of the users could not answered the questionnaires.
However, seniors' opinion about robots have been compared with their previous \gls{thri} with children.
The younger related to robots as pets or servants.
On the other hand, elderly related them as companions or friends.
This evidence might suggest a feeling of loliness and the lack of company.
\todo{Can this Dautenhahn research belong to this section? Or should I find a way to include it in the related work?}

% necessidade de perceber se sao aceites quer por utilidade quer por companhia e fazer referenca a heerink evers e wielinga